\documentclass[11pt,a4paper]{article}
\usepackage[utf8]{inputenc}
\usepackage[czech]{babel}
\usepackage[T1]{fontenc}
\usepackage{amsmath}
\usepackage{amsfonts}
\usepackage{amssymb}
\usepackage[left=2cm,right=2cm,top=2.5cm,bottom=2cm]{geometry}
\author{Stanislav Nechutný}
\begin{document}
Stanislav Nechutný (xnechu01)

1. projekt IPK

\section{Protokol}
	Pro komunikaci byla zvolena kombinace binárního protokolu s textovým, kdy prvních 8 bitů je číslice reprezentující formát přenášených dat. V závislosti na tomto formátu je následný zbytek zprávy načten a zpracován.
	
	\subsection{Komunikace}
	Komunikace je zahájena klientem odesláním paketu s typovým číslem 0 a očekává od serveru odpověď s kódem 1.
	
	Následně je zaslána další část s kódem 2 následovaná polem hodnot enumu s hodnotami reprezentující parametry přenosu - jaké hodnoty z /etc/passwd jsou požadovány a v jakém pořadí. Server na přijetí těchto dat odpoví kódem 7, značícím "OK" - tedy, že je připraven přijímat požadavky a zasílat data.
	
	Klient začne zasílat jednotlivé požadavky s kódem 3 (login), nebo 4 (UID) a v datové části je vždy umístěno vyhledávané uživatelské jméno, nebo číslo UID. Server při přijetí provede vyhledání a v závislosti na úspěšnosti nalezení dat odesílá kód 5 s řetězcem obsahující pouze požadované údaje, nebo kód 10 s popisem chyby.
	
	Poté, co klient přijme všechna dotázaná data a již nemá zájem o další odešle kód 6 - End a ukončí spojení.
	
	
\end{document}